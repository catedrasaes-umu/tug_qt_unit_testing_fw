
\subsection{Library Utils for Writing Tests}

As said above, {\tt qt\_TestUtils} includes a set of methods aimed at
supporting the definition of new tests. These methods can be accessed
through the namespace {\tt tug::}.
%
Some of these methods are described next.

\vspace{1ex}

Launch and destroy panels:
%
\begin{lstlisting}
// launches a panel/window object
void panel_launch(QWidget* panel)
void panel_launch(QWidget& panel)

// deletes a panel/window object
void panel_destroy(QWidget* panel)
void panel_destroy(QWidget& panel)
\end{lstlisting}

\begin{lstlisting}
Example:
----------------------------------------------------------------------
test_panel panel;
tug::panel_launch(panel);
\end{lstlisting}


Assertions:
%
\begin{lstlisting}
// checks a boolean expression
void assert(bool expr)

// checks a boolean expression and displays 'msg' if error
void assert_msg(bool expr, const char* msg)

// prints a warning message in a test
void warning(const char* msg)

// simulates an error in a test
void fail(const char* msg)
\end{lstlisting}

\begin{lstlisting}
Example:
----------------------------------------------------------------------
tug::assert(1 == true);
tug::assert_msg(1 == false,``This will never be true'');
tug::warning(``This is a warning message'');
tug::fail(``Simulating a failure'');
\end{lstlisting}


Update/repaint panels:
%
\begin{lstlisting}
// repaints a panel
void panel_repaint(QWidget* panel)
void panel_repaint(QWidget& panel)

// hides and shows a panel. It forces repaint and update.
void panel_blink(QWidget* panel)
void panel_blink(QWidget& panel)
\end{lstlisting}

\begin{lstlisting}
Example:
----------------------------------------------------------------------
test_panel panel; tug::panel_launch(panel);
// do something here...
tug::panel_repaint(panel); //this repaints the user interface
// do something more...
tug::panel_blink(panel); //this forces a panel update
\end{lstlisting}


Sleeps:
%
\begin{lstlisting}
// sleeps 'ms' milliseconds
void sleep(int ms)

// sleeps 1 second
void sleep1()

// sleeps 2 second
void sleep2()

// sleeps 3 second
void sleep3()

// sleeps 5 second
void sleep5()
\end{lstlisting}

\begin{lstlisting}
Example:
----------------------------------------------------------------------
tug::sleep2();
\end{lstlisting}


Random values:
%
\begin{lstlisting}
// resets random numbers generator
void random_reset()

// generates a random number between 0 and 'n'
int random(int n)

// generates a random number between 'low' and 'high'
int random_in_range(int low, int high)

// generates true or false randomly
bool random_bool()
\end{lstlisting}

\begin{lstlisting}
Example:
----------------------------------------------------------------------
tug::random_reset();
panel->amethod_expecting_int(tug::random(10));
panel->amethod_expecting_percentage(tug::random(0,100));
panel->amethod_expecting_bool(tug::random_bool());
\end{lstlisting}


Simulation of segmentation faults:
%
\begin{lstlisting}
// simulates a segmentation fault
void segfault()
\end{lstlisting}

\begin{lstlisting}
Example:
----------------------------------------------------------------------
tug::segfault(); //at this point a testsuite ends its execution
\end{lstlisting}



Timers:
%
\begin{lstlisting}
/// starts timer
void timer_start()

/// returns milliseconds elapsed from last call to timer_start
long timer_elapsed_ms()
\end{lstlisting}

\begin{lstlisting}
Example:
----------------------------------------------------------------------
tug::timer_start();

tug::sleep1();
tug::log() << "timer 1: " << tug::timer_elapsed_ms();
tug::sleep2();
tug::log() << "timer 2: " << tug::timer_elapsed_ms();

tug::timer_start(); //timer reset

tug::sleep2();
tug::log() << "timer 3: " << tug::timer_elapsed_ms();
\end{lstlisting}


Ranges:
%
\begin{lstlisting}
/// ForRange
template <typename T>
class ForRange(T l, T u) 

/// ForNestedRange
template <typename T>
class ForNestedRange(T l, T u, T li, T ui)
\end{lstlisting}

\begin{lstlisting}
Example:
----------------------------------------------------------------------
tug::ForRange<double>(-10,70)
   .call<test_panel>(_panel, &test_panel::set_sbLatMin)
   .call_void<test_panel>(_panel, &test_panel::doClick_btApplyCenter)
   .repaint(_panel) //optional
   .sleep_ms(100)   //optional - default is 0
   .increment(2)    //optional - default is 1
   .run();

tug::ForNestedRange<int>(-100,100,-200,200)
   .call<test_panel>(_panel, &test_panel::set_sbLatDegrees)
   .call_inner<test_panel>(_panel, &test_panel::set_sbLongDegrees)
   .call_void<test_panel>(_panel, &test_panel::doClick_btApplyCenter)
   .repaint(_panel) //optional
   .increment(5)    //optional - default is 1
   .run();
\end{lstlisting}


Log:
%
\begin{lstlisting}
// log adds a log line to a test
void log(const char* s)
\end{lstlisting}

\begin{lstlisting}
Example:
----------------------------------------------------------------------
tug::log(``log a sentence...'');
tug::log() << ``or use it as `` << 1 << `` stream.'';
\end{lstlisting}


Macros for value ranges (out of {\tt tug::} namespace):
%
\begin{lstlisting}
// This macro repeates a code 'n' times. 
// Additionally, values from 0 to 'n-1' are generated.
// 'value' is the name of the variable to be used.
tug__REPEAT(n)

// This macro simulates an integer range between 'min' and 'max', both
// included.
// 'value' is the name of the variable to be used.
tug__INT_RANGE(min,max)

// This macro simulates an integer range between 'min' and 'max', both
// included, using 'inc' as increment.
// 'value' is the name of the variable to be used.
tug__INT_RANGE_INC(min,max,inc)

// This macro simulates a float range between 'min' and 'max', both
// included, using 'inc' as increment.
// 'value' is the name of the variable to be used.
tug__FLOAT_RANGE(min,max,inc)

// This macro executes a code 'n' iterations. 
// At each iteration 'value' holds a random value between 0 
// and 'random_limit'.
// 'value' is the name of the variable to be used.
tug__RANDOM_INT_SET(n,random_limit)

// This macro executes a code 'n' iterations.
// At each iteration 'value' holds a random value true or false.
// 'value' is the name of the variable to be used.
tug__RANDOM_BOOL_SET(n)
\end{lstlisting}









%%% Local variables:
%%% mode: latex
%%% TeX-master: "README.tex"
%%% ispell-local-dictionary: "american"
%%% coding: utf-8
%%% fill-column: 75
%%% TeX-parse-self: t
%%% TeX-auto-save: t
%%% End:
