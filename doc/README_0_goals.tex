
\section{Motivation}

\begin{flushright}
{\it TUG --- ``GUI Unit Testing'' --- ``Testing Unitario GUI'', in Spanish.}
\end{flushright}

TUG project~\footnote{The TUG Project is an initiative of C\'atedra
  SAES (\url{http://www.catedrasaes.org}) funded by the SAES company
  (\url{http://www.electronica-submarina.com}). This project and all
  its components have been designed and developed at University of
  Murcia (Spain).}
%
was born with the main purpose of providing a unit testing framework
for graphical user interfaces. The main goal was providing developers
with a method to easily create a battery of tests for Qt-based
applications. The tests had to simulate, as far as possible, users
interaction with the interface.

With this purpose, the TUG project is divided into two main components:
\begin{itemize}
%
\item {\bf TUG Wizard}: a wizard-like application that helps developers to
  create and configure, step by step, a test project aimed at testing a Qt
  based panel as well as the underlying model and communication classes (if
  they exists). It generates a new panel inheriting the original one. This
  new panel includes customized methods to simulate users interaction with
  the widgets composing the panel. It can also generate a full, standalone
  test project including testsuites and empty test methods ready for being
  filled with testing code.
%
\item {\bf TUG Base Library}: a library aimed at supporting the tests
  generated by TUG Wizard, as well as test projects created manually by
  developers. It provides a way to structure test suites, as well as a set
  of methods to support the definition of GUI tests, all around the Qt
  Test framework.
%
\end{itemize}

Along this document it is described how to properly use these
components to test your Qt-based projects.

%%% Local variables:
%%% mode: latex
%%% TeX-master: "README.tex"
%%% ispell-local-dictionary: "american"
%%% coding: utf-8
%%% fill-column: 75
%%% TeX-parse-self: t
%%% TeX-auto-save: t
%%% End:
